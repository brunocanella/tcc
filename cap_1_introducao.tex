\chapter{Introdução}

Desde seu surgimento, popularização e evolução até os dias atuais, os veículos automotivos aumentaram em muito a sua complexidade, a ponto de que apenas o conhecimento mecânico do veículo não é mais suficiente. A quantidade de componentes eletrônicos presentes nos veículos automotivos aumentou consideravelmente passando, inclusive, a substituir sistemas puramente mecânicos. Dentre os fatores que alavancaram estas mudanças destacam-se o barateamento, miniaturização e popularização dos componentes eletrônicos, bem como a adequação da indústria automobilística as novas legislações, que passaram a ditar a emissão máxima de poluentes e a exigir mecanismos de segurança, como o \criarSigla[Sistema de Anti-Bloqueio]{Antiblockier-Bremssystem}{ABS} e o \emph{Airbag}.

Este aumento de componentes eletrônicos nos automóveis passou a exigir, também, um maior número de \criarSigla[Unidades de Controle Eletrônica]{Eletronic Control Units}{ECU} para realizar seu gerenciamento, que passaram a ser espalhadas pelo chassi do veículo conforme a proximidade ao sistema tal qual gerenciam. Com o grande número de desenvolvedores de sistemas para veículos e, consequentemente, de ECUs, surgiu um novo desafio no mercado, de garantir que estas centrais pudessem se comunicar umas com as outras, bem como serem diagnosticadas quanto a presença de erros, como quando um sistema restrito temporalmente venha a falhar.

Com o propósito de padronizar e, assim, facilitar o desenvolvimento e intercambialidade de auto-peças por terceiros, as principais montadoras e fabricantes de veículos entraram em um consenso, estipulando um padrão de normas de desenvolvimento para veículos automotivos, chamada de \criarSigla{AUTomotive Open System ARchitecture}{AUTOSAR}, a qual se encontra em sua quarta revisão.

Para gerenciar os diversos módulos eletrônicos agora presentes em um veículo, bem como garantir a interoperabilidade entre eles, foram criadas normas referente ao desenvolvimento de sistemas operacionais embarcados. Estas normas visam o estabelecimento de padrões para o funcionamento, comunicação e especificações do \criarSigla{sistema operacional}{SO}, sem sacrificar a liberdade criativa de desenvolvimento do sistema, como a seleção de hardwares e implementação de algoritmos.

\section{Justificativa e Motivação}

A maioria das soluções em SOs automotivos são exclusivamente comerciais e de código fechado. Embora existam soluções de código aberto para sistemas embarcados, não existe, na atualidade, um SO de código aberto homologado nos padrões do AUTOSAR. O projeto que mais se aproxima deste cenário é o Trampoline, que se encontra em fase de homologação pelo consórcio AUTOSAR \cite{Trampoline:HOME}.

Visando a criação de um SO embarcado que mantivesse um modelo de desenvolvimento de código aberto, surgiu a idealização do \criarSigla[Sistema Operacional Automotivo Aberto]{Open AUTomotive Operating System}{OpenAUTOS}. Através do desenvolvimento do OpenAUTOS, deseja-se alcançar um SO nacional que seja referência na área, utilizando componentes e tecnologias com alta disponibilidade e de fácil acesso, além de agregar contribuições com a própria comunidade acadêmica.

\section{OBJETIVOS}

Esta seção apresenta o objetivo geral e os objetivos específicos deste trabalho.

\subsection{Geral}

Desenvolver um sistema operacional embarcado de código aberto que atenda as normas estabelecidas pelo padrão AUTOSAR.

\subsection{Específicos}
\begin{enumerate}
	\item Levantar o estado da arte com respeito a algoritmos para sistemas operacionais embarcados;
	\item Estudar padrões de sistemas automotivos;
	\item Levantar os requisitos para implementação de um SO de acordo com a norma AUTOSAR;
	\item Estabelecer um projeto de código aberto em um repositório online;
	\item Documentar o código do projeto;
	\item Avaliar o sistema desenvolvido.
\end{enumerate}

\section{Organização do trabalho}

Este trabalho está dividido em 6 capítulos, contando com a introdução, além de 1 apêndice e 3 anexos.

O \textbf{Capítulo 2} apresentará os domínios eletrônicos de funcionamento em veículos, seus meios de comunicação, unidades de controle e engenharia de software automotivo.

O \textbf{Capítulo 3} abordará os principais conceitos sobre sistemas operacionais. Ao final do capítulo serão relatados alguns estudos de caso a respeito de sistemas operacionais embarcados com foco para a automação veicular.

O \textbf{Capítulo 4} apresentará a proposta do SO OpenAUTOS, destacando as escolhas tanto do projeto de software bem como a arquitetura de hardware adotada.

O \textbf{Capítulo 5} descreverá os testes e resultados de validação do OpenAUTOS, discutindo o comportamento do SO quanto ao que é esperado pela norma.

O \textbf{Capítulo 6} apresentará as considerações finais, bem como sugestões para trabalhos futuros.

O apêndice apresenta os códigos utilizados para a realização dos experimentos do OpenAUTOS.

Os anexos apresentam as interfaces de código do SOE OpenAUTOS divididos em 3 partes: Interfaces do OSEK/VDX, do OpenAUTOS OS e para a plataforma alvo.
