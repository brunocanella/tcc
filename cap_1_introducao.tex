\chapter{Introdução}

Desde seu surgimento, popularização e evolução até os dias atuais, os veículos automotivos aumentaram em muito a sua complexidade, a ponto de que apenas o conhecimento mecânico do veículo não é mais suficiente. A quantidade de componentes eletrônicos presentes nos veículos automotivos aumentou consideravelmente passando, inclusive, a substituir sistemas puramente mecânicos. Fatores que alavancaram estas mudanças incluem o barateamento, miniaturização e popularização dos componentes eletrônicos.

Com o propósito de padronizar e, assim, facilitar o desenvolvimento e intercambialidade de auto-peças por terceiros, as principais montadoras e fabricantes de veículos entraram em um consenso e estipularam um padrão de normas de desenvolvimento para veículos automotivos, chamada de \emph{AUTOSAR}, a qual se encontra em sua quarta revisão.

Para gerenciar os diversos módulos eletrônicos agora presentes em um veículo, bem como garantir a interoperabilidade deles, foi criada uma norma referente ao desenvolvimento de sistemas operacionais. Esta norma visa o estabelecimento de padrões para o funcionamento, comunicação e especificações do sistema, sem sacrificar a liberdade criativa de desenvolvimento do sistema, como a seleção de hardwares e implementação de algoritmos.

%Um dos padrões que surgiu para auxiliar nesta integração foi referente ao desenvolvimento de sistemas operacionais, responsável por gerenciar todos os recursos diretamente relacionados ao funcionamento do veículo. Esta norma visa o estabelecimento de padrões para o funcionamento, comunicação e especificações do sistema, sem sacrificar a liberdade criativa de desenvolvimento do sistema, como a seleção de hardwares e implementação de algoritmos.

\section{Justificativa e Motivação}

A maioria das soluções em \emph{SO}s automotivos são exclusivamente comerciais e de código privado. Embora existam soluções de código aberto para sistemas embarcados, não existe, na atualidade, um \emph{SO} de código aberto homologado nos padrões do \emph{AUTOSAR}. O projeto que mais chega próximo deste cenário é o \emph{Trampoline}, que se encontra em fase de homologação da norma.

Visando a criação de um \emph{SO} embarcado, para uso em veículos populares, e que mantivesse um padrão de código aberto, surgiu a idealização do \emph{OpenAUTOS}. Com o desenvolvimento do \emph{OpenAUTOS} é desejado conseguir um modelo de SO nacional, com componentes e tecnologia disponíveis de fácil acesso, proporcionando também contribuir com a própria comunidade acadêmica.

\section{OBJETIVOS}

Esta seção apresenta o objetivo geral e os objetivos específicos deste trabalho.

\subsection{Geral}

Desenvolver um sistema operacional embarcado de código aberto que atenda as normas estabelecidas pelo padrão \emph{AUTOSAR}.

\subsection{Específicos}
\begin{enumerate}
	\item Levantar o estado da arte com respeito a algoritmos para sistemas operacionais embarcados;
	\item Estudar padrões de sistemas automotivos;
	\item Levantar os requisitos para implementação de um \emph{SO} de acordo com a norma \emph{AUTOSAR};
	\item Estabelecer um projeto de código aberto em um repositório online;
	\item Documentar o código do projeto;
	\item Criar um modelo físico que utilize o \emph{SO} desenvolvido;
	\item Realizar a instalação do \emph{SO} em um veículo automotivo real;
\end{enumerate}

\section{Organização do trabalho}

Este trabalho está dividido em 4 capítulos, contando com a introdução.

O \textbf{Capítulo 2} apresentará os domínios eletrônicos de funcionamento em veículos, seus meios de comunicação, unidades de controle e engenharia de software automotivo.

O \textbf{Capítulo 3} abordará os principais conceitos sobre sistemas operacionais. Ao final do capítulo serão relatados alguns estudos de caso a respeito de sistemas operacionais embarcados com foco para a automação veicular.

O \textbf{Capítulo 4} apresentará a proposta do SO \emph{OpenAUTOS}, destacando as escolhas e a arquitetura adotada.
