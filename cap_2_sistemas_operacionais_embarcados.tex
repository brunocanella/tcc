\chapter{Sistemas Operacionais Embarcados}

Sendo o objetivo do trabalho a construção de um \emph{Sistemas operacionais embarcados},  são um dos pontos principais deste trabalho e 

\section{Sistemas Operacionais}

Em seu livro sobre sistemas operacionais, \citeonline{minix} explica que um computador é composto por um ou mais processadores, uma memória principal e dispositivos de entrada e saída. O sistema operacional tem por objetivo controlar os recursos do computador e prover a base na qual as aplicações podem ser escritas. Esta base provê uma interface que provê para o programador de aplicativos um meio de acesso a funções de mais baixo nível e hardware.

Ele também menciona que definir um sistema operacional é uma tarefa difícil, e que parte do problema se deve ao fato de ele realizar duas funções distintas: expandir a máquina e gerenciar recursos.

Expandir a máquina se refere á abstração feita sobre o seu hardware, escondendo as complexidades e facilitando a sua manipulação através de uma interface de mais alto nível. 

Gerenciar recursos se refere ao compartilhamento destes em tempo e espaço. Em tempo pode ser exemplificado por um sistema que precisa executar diversas tarefas ao mesmo tempo em um computador mas onde apenas uma delas pode estar ativa. Neste caso, o sistema operacional alterna em dados intervalos de tempo qual processo estará sendo executado. Impedir o acesso á áreas de memória reservada ou ainda permitir que um processo aloque a impressora são exemplos da gerenciamento de recursos em espaço.



%\textoResumo{Este é um texto de resumo}

Breve introdução

Realizam duas tarefas distintas: expandir a máquina e gerenciar recursos.

\section{Sistemas Embarcados}

\section{Sistemas Operacionais Embarcados}


*Definição de Sistemas Embarcados e Sistemas Operacionais Embarcados

\section{Estrutura de um Sistema Operacional Embarcado}

adsdasds

\section{Padrões para o Desenvolvimento de RTOS (Sistemas Operacionais Embarcados de Tempo Real) }

asdasdas

\section{Exemplos}

asdasdsada

