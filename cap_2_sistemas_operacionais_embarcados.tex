\chapter{Sistemas Operacionais Embarcados}

O presente capítulo inicia abordando o tema de sistemas operacionais e sobre como estes atuam em ambientes computacionais, mantendo um foco mais voltado aos de uso pessoal.

%TODO falar sobre Associar veiculos a sistemas embarcados
Na sequência, há uma breve discussão sobre o que são sistemas embarcados, onde podemos encontrá-los  e como podemos classifica-los. Uma discussão sobre as funcionalidades que um sistema embarcado pode agregar aos sistemas veiculares é realizada no capítulo . %\ref{cap:automacao_veicular}.

Definidos estes conceitos, um modelo de estrutura para um sistema operacional é apresentado, mostrando e detalhando os principais componentes que o compõem.

Em seguida, o conceito de sistemas de tempo real é agregado ao trabalho, apresentando uma definição para o mesmo, bem como exemplificando casos que justifiquem a sua associação para com projetos veiculares. O conceito de sistemas operacionais embarcados é então conformado para um padrão que atenda as especificações de um projeto com restrições temporais.

Por fim, são apresentados exemplos de sistemas operacionais em tempo real, buscando destacar o que eles apresentam em termos de inovação ou funcionalidade.

\section{Sistemas Operacionais}

Em seu livro sobre sistemas operacionais, \citeonline{minix} explica que um computador é composto por um ou mais processadores, uma memória principal e dispositivos de entrada e saída. O sistema operacional tem por objetivo controlar os recursos do computador e prover a base na qual as aplicações podem ser escritas. Esta base disponibiliza uma interface para o programador de aplicativos, que atua como um meio de acesso a funções de mais baixo nível e ao hardware.

Ele também menciona que definir um sistema operacional é uma tarefa difícil, e que parte do problema se deve ao fato de ele realizar duas funções distintas: expandir a máquina e gerenciar recursos.

Expandir a máquina se refere á abstração feita sobre o seu hardware, escondendo as complexidades e facilitando a sua manipulação através de uma interface de mais alto nível. 

Gerenciar recursos se refere ao compartilhamento destes em tempo e espaço. Em tempo pode ser exemplificado por um sistema que precisa executar diversas tarefas ao mesmo tempo em um computador mas onde apenas uma delas pode estar ativa. Neste caso, o sistema operacional alterna em dados intervalos de tempo qual processo estará sendo executado. Impedir o acesso á áreas de memória reservada ou ainda permitir que mais de um processo aloque a impressora são exemplos da gerenciamento de recursos em espaço.

\section{Sistemas Embarcados}

De forma objetiva, \citeonline{prog_emb_sys} definem um sistema embarcado como uma combinação de hardwares e softwares computacionais em possível conjunto com partes mecânicas e/ou eletrônicas, integrados para realizar uma função dedicada.

\citeonline{hallinan} expande esta definição descrevendo uma série de características que ajudam a classificar quando um sistema pode ser considerado embarcado, sendo elas:

\begin{itemize}
	\item Possuir um mecanismo de processamento, tal como um microcontrolador de uso geral;
	\item Tipicamente projetados para uma aplicação ou propósito específico;
	\item Oferecem, opcionalmente, uma interface de usuário simples como, por exemplo, um controle de ignição para um motor automotivo;
	\item Frequentemente possuem recursos limitados. A exemplo, eles podem ter uma memória principal pequena e nenhum disco-rígido;
	\item Podem ter um suprimento limitado de energia, como em um sistema alimentado por baterias;
	\item Tipicamente, não são utilizados como plataformas computacionais de propósito geral;
	\item Geralmente possuem aplicativos de software pré-instalados, sem seleção do usuário;
	\item São distribuídos com todas as aplicações de software e hardware pré-integrados;
	\item Comumente são projetados para aplicações onde não há intervenção humana;
\end{itemize}

Na maioria dos casos, os sistemas embarcados são limitados em recursos quando comparados ao tipico computador de uso geral. Os sistemas embarcados geralmente possuem memória limitada, pequeno ou nenhum espaço de disco rígido e, algumas vezes, nenhuma forma de conectividade com redes externas. Frequentemente, as únicas interfaces com o usuário são uma porta serial e alguns LEDs\footnote{Light Emitting Diode ou Diodo Emissor de Luz}\cite{hallinan}.

\citeonline{qingli} fala que sistemas embarcados são encontrados em uma variedade infinita de tipos, cada qual exibindo caracteristicas únicas. A exemplo, ele cita que a maioria dos veículos em transito, hoje em dia, imbuem chips de computadores, os quais executam tarefas que agregam funcionalidades aos veículos, tornando-os mais fáceis, objetivos, seguros e agradáveis de dirigir. Outros exemplos seriam: telefones, casas inteligentes, sistemas de segurança, aparelhos de televisão a cabo e a satélite, sistemas de \emph{home theater}, secretária eletrônica, dentre muitos outros.

\section{Componentes de um Sistema Operacional Embarcado}

%TODO Escrever esta parte
Paradas que compõem um \emph{kernel}. O conteúdo que segue foi adaptado do livro de \citeonline{qingli}. Informações provenientes de outros autores estarão devidamente citados.

\subsection{Gerenciamento de Tarefas}

Tarefas são threads de execução independentes que competem entrem si para se tornarem o processo ativo no processador. As aplicações são decompostas em múltiplas tarefas concorrentes para optimizar a manipulação de entradas e saídas atendendo a uma restrição temporal.

Uma tarefa precisa ser agendável. Ela deve ser capaz de competir pelo tempo de execução em um sistema, baseado em um algoritmo de agendamento, como \emph{round-robin} ou por prioridades. Uma tarefa é definida pelo seu distinto conjunto de parâmetros e estruturas de dado que a suportam. Ao ser criada, uma tarefa recebe um identificador único, uma prioridade, uma estrutura de contexto e uma rotina que será executada sempre que a tarefa estiver em execução. Um exemplo de estrutura para uma tarefa pode ser visto no algoritmo \ref{alg:struct_task}.

\begin{listing}
	\label{alg:struct_task}
	\caption{Estrutura de uma Tarefa.}
	\centering
	\begin{minted}[mathescape,linenos,numbersep=5pt,gobble=0,frame=lines,framesep=2mm]{c++}
typedef struct task_s {
	int id;
	int priority;
	state_t state;
	context_t context;
	void (*function)();	
} task_t;
	\end{minted}
\end{listing}

Além de prover a estrutura para um objeto do tipo \emph{tarefa}, o \emph{kernel} também é responsável por oferecer serviços que manipulam estes objetos. Estes serviços incluem ações que o kernel realiza sigilosamente para suportar as tarefas como, por exemplo, criar e manter o bloco de controle e pilha de uma tarefa.

O \emph{kernel} também deve prover uma API\footnote{Application Programming Interface - Interface de Programação de Aplicação} que permita ao desenvolvedor manipular as tarefas. Algumas das operações mais comuns que os desenvolvedores podem realizar em uma tarefa incluem:

\begin{itemize}
	\item Criar e excluir;
	\item Controlar o agendamento;
	\item Recuperar informações.
\end{itemize}

Uma tarefa pode ser destinada a realizar tanto uma atividade do sistema quanto uma funcionalidade implementada pelo usuário final. Qualquer que seja seu propósito, uma tarefa possuí um conjunto de estados, os quais definem o seu comportamento no sistema. Dentre os mais comuns estão os estados de: \emph{pronto}, \emph{rodando} e \emph{bloqueado}.

\begin{itemize}
	\item \textbf{Pronto}: A tarefa está pronta para execução e está a espera do escalonador de processos;
	\item \textbf{Rodando}: A tarefa foi escolhida pelo escalonador e está efetivamente em execução;
	\item \textbf{Bloqueado}: A tarefa está aguardando a liberação de algum recurso antes que possa prosseguir sua execução.
\end{itemize}

\emph{Pronto} é  sempre o estado inicial de uma tarefa. Dependendo do tipo de tarefa, ela pode executar em um looping infinito ou até que ela encerre. Neste ultimo caso, a tarefa é removida da lista de processos aptos. Uma tarefa só pode mudar seu estado para bloqueado quando ela for a tarefa em execução. Uma tarefa que está bloqueada, e cujo recurso foi liberado, pode apenas retornar para a fila de espera e aguardar que o escalonador a coloque em execução novamente.

A figura \ref{fig:tarefa_estados} apresenta uma ilustração das transições de estado que uma tarefa pode fazer.

\begin{figure}[htb]
	\centering
	\includegraphics[width=\textwidth]{task_states.png}
	\caption{Transição de Estados para Tarefas.}
	\label{fig:tarefa_estados}
\end{figure}

\subsubsection{Fibra}

Uma variante das tarefas são as fibras\footnote{Fiber}, que são threads de execução leves e não-preemptivas, normalmente utilizadas para executar porções de código responsáveis pelos dispositivos de drivers e outros trabalhos de performance crítica \cite{rocket}.

\subsubsection{Semáforos}

Múltiplas \emph{threads} de execução concorrentes em uma aplicação devem serem capazes de sincronizar suas execuções e coordenar o acesso mutuamente exclusivo a recursos compartilhados. Para atender a estes requisitos, os \emph{kernels} provêm um objeto chamado semáforo e respectivos serviços associados a este.

Um semáforo é um objeto do \emph{kernel} que uma ou mais threads de execução podem reservar ou liberar, para o propósito de sincronização ou exclusão mútua. Os elementos que compõem um semáforo são: um identificador único, um contador e uma fila de tarefas em espera. Um exemplo de estrutura de um semáforo pode ser visto no algoritmo \ref{alg:struct_semaphore}.

\begin{listing}
	\label{alg:struct_semaphore}
	\caption{Estrutura de uma Semáforo.}
	\centering
	\begin{minted}[mathescape,linenos,numbersep=5pt,gobble=0,frame=lines,framesep=2mm]{c++}
	typedef struct semaphore_s {
	int id;
	int value;
	queue_t waiting_tasks;
	} semaphore_t;
	\end{minted}
\end{listing}

Um semáforo funciona como uma espécie de chave que permite a uma tarefa acessar algum tipo de operação ou recurso. Se a tarefa puder adquirir um semáforo, ela poderá continuar sua execução normalmente.

Um semáforo pode ser reservador um número finito de vezes. Uma vez que seu valor máximo seja alcançado, as tarefas que tentarem adquirir este semáforo serão enviadas para uma lista de espera, onde permanecerão até que uma tarefa que tenha alocado previamente o semáforo o libere. Assim que se torna disponível, o semáforo é realocado para a primeira tarefa na fila de espera, caso haja alguma. Do contrário, seu valor re-incrementa.

Os semáforos aqui apresentados são os de contagem, binário e de exclusão mútua. Semáforos de contagem e binários apresentam comportamentos similares e respeitam as restrições impostas anteriormente, tendo como única diferença que um semáforo binário possui seu valor máximo igual a $1$. Os diagramas de estados destes dois modelos podem ser visualizados nas figuras \ref{fig:semaphore_binary} e \ref{fig:semaphore_counting}.

\begin{figure}[htb]
	\centering
	\includegraphics[width=\textwidth]{semaphore_binary.png}
	\caption{Transição de Estados para Semáforos Binários.}
	\label{fig:semaphore_binary}
\end{figure}

\begin{figure}[htb]
	\centering
	\includegraphics[width=\textwidth]{semaphore_counting.png}
	\caption{Transição de Estados para Semáforos Numéricos.}
	\label{fig:semaphore_counting}
\end{figure}

\subsubsection{Mutex}

Um terceiro tipo de semáforo trata de um caso especial do semáforo binário, chamado de \emph{semáforo de exclusão mútua} ou \emph{mutex}.  Um \emph{mutex} pode suportar propriedades de posse, trava recursiva, deleção segura de tarefas, dentre outros protocolos para evitar problemas inerentes a exclusão mútua.

\pagebreak

\begin{itemize}
%	\setlength\itemsep{1mm}
	\item \textbf{Posse}: Quando uma tarefa aloca um \emph{mutex}, ela se torna seu proprietário até que ela libere este recurso. Contrário aos semáforos, onde qualquer tarefa pode liberar o recurso, inclusive uma tarefa que não alocou o semáforo, apenas a tarefa proprietária poderá liberar o \emph{mutex}.
	\item \textbf{Trava Recursiva}: Permite a tarefa que alocou o \emph{mutex} realizar chamadas adicionais para trava-lo novamente. Um contador interno é utilizo para manter o registro da quantidade de vezes que o recurso foi alocado para a tarefa proprietária. O \emph{mutex} só volta a ser liberado quando a tarefa proprietária chama um número igual de vezes a rotina de liberação deste recurso.
	\item \textbf{Deleção segura}: A deleção prematura de uma tarefa pode ser evitada utilizando uma trava de deleção para quando a tarefa trava e destrava um \emph{mutex}. Ativar esta propriedade garante que uma tarefa que for proprietária do \emph{mutex} não possa ser excluída até que o mesmo seja liberado.
	\item \textbf{Anulação de Inversão de Prioridade}: Dado o cenário onde uma tarefa de prioridade inferior que aloca um \emph{mutex}, e onde uma tarefa de prioridade maior está aguardando que ele seja liberado, mas que está sendo ofuscada por uma terceira tarefa de prioridade mediana, ocorre o que é chamado de \emph{Inversão de Prioridade}, pois a tarefa mais prioritária passaria a ser aquela que alocou o \emph{mutex}. Algumas técnicas para evitar estas situações são: elevar a prioridade da tarefa que aloca o \emph{mutex} ao máximo ou ainda por herança de prioridade da tarefa que ficar bloqueada por aguardar o \emph{mutex}.
\end{itemize}

Contrário aos estados de \emph{disponivel}/\emph{indisponivel} dos semáforos de contagem/binários, os estados de um \emph{mutex} são \emph{destravado} e \emph{travado}, representados por $0$ e $1$ respectivamente. Um \emph{mutex} é inicialmente criado em seu estado \emph{destravado}, no qual ele pode ser reservado por uma tarefa. Após a alocação, o \emph{mutex} passa para o estado \emph{travado}. Reciprocamente, quando a tarefa libera o \emph{mutex}, este retorna para o estado \emph{destravado}. A figura \ref{fig:semaphore_mutex} mostra o diagrama de estados para um \emph{mutex} com e sem travamento recursivo.

\begin{figure}[htb]
	\centering
	\includegraphics[width=\textwidth]{semaphore_counting.png}
	\caption{Transição de Estados para Semáforos Numéricos.}
	\label{fig:semaphore_mutex}
\end{figure}

\subsubsection{Filas de Mensagens}

Uma fila de mensagens é um objeto que possui buffer através do qual tarefas e ISRs\footnote{Interrupt Service Routines - Rotinas de Serviço para Exceções) enviam e recebem mensagens para comunicação e sincronizar com dados. Uma fila de mensagens é uma espécie de pipeline. Ela armazena temporariamente as mensagem do remetente até que o destinatário esteja pronto para recebe-las. Esta bufferizaçao temporária desacopla uma tarefa emissora e receptora; isto é, ela libera a tarefa de ter que enviar e receber mensagens simultaneamente.

Assim como os semáforos, uma fila de mensagens tem diversos componentes que o \emph{kernel} utiliza para gerenciar a fila. Quando uma fila de mensagens é criada, ela recebe um QCB\footnote{Queue Control Block - Bloco de Controle de Fila} associado, um identificar único, buffers de memória, um tamanho de fila, um limite máximo de mensagens, e uma ou mais listas de tarefas aguardando.

\begin{listing}
	\label{alg:struct_semaphore}
	\caption{Estrutura de uma Semáforo.}
	\centering
	\begin{minted}[mathescape,linenos,numbersep=5pt,gobble=0,frame=lines,framesep=2mm]{c++}
typedef struct message_queue_s {
	int id;
	int queue_size;
	int queue_max;
	queue_t waiting_tasks;
} message_queue_t;
	\end{minted}
\end{listing}

\subsection{Pipes}

\subsection{Eventos}
\subsection{Sinais}
\subsection{Variáveis Condicionais}

\subsection{Exceções e Interrupções}

\subsection{Temporizadores}

\subsection{Subsistemas de Entrada/Saída}

\subsection{Gerenciamento de Memória}

\subsection{Sistemas de Arquivos}


\section{Sistema de Tempo Real}

A real-time application is composed of multiple concurrent tasks that are independent threads of execution, competing on their own for processor execution time \cite{qingli}.

\section{Padrões para o Desenvolvimento de RTOS}

\section{Exemplos}
